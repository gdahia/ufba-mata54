\documentclass[12pt]{article}

\usepackage[utf8]{inputenc}
\usepackage{amsmath}

\hyphenation{an-te-rior-men-te}

\title{Segundo trabalho prático - MATA54}

\author{Gabriel Dahia}

\begin{document}

  \maketitle

  \section{Considerações gerais}
    Este documento descreve as estruturas de dados e algoritmos utilizados para a implementação do segundo trabalho prático da disciplina MATA54 - Estrutura de Dados e Algoritmos II da Universidade Federal da Bahia.

    Durante todo o texto, a não ser que indicado, $n$ se refere ao número total de palavras na base de palavras e $m$ se refere ao maior comprimento possível de uma palavra.
    Quando $s$ é a palavra da operação de digitação de número $i$ e $p$ é a palavra da operação de digitação número $i + 1$, é dito que $p$ \emph{sucede} $s$.

    Foram feitas as seguintes suposições sobre o funcionamento do sistema:
    \begin{itemize}
      \item Caso uma palavra esteja no banco de palavras, ela não será reinserida através de uma operação de inserção;
      \item O término de uma execução do programa significa o fim do texto digitado.
        Isto é, a primeira palavra digitada em uma execução do programa não sucede a última palavra digitada na execução anterior;
      \item Se uma palavra $p$ é prefixo de outra palavra $s$, então $p < s$, onde $<$ é a ordem alfabética;
      \item Não há limite para o número de arquivos criados.
    \end{itemize}

    \subsection{Trie}
    \emph{Tries} são implementadas da seguinte maneira: cada nó possui uma lista de índices e uma lista de pares $<c, u>$, onde $c$ é um caractere e $u$ é um nó de trie.
    Se em um nó $v$ há um par $<c, u>$, então existe a transição de $v$ para $u$ mediante a leitura de $c$. 
    No contexto de tries, $r_T$ é a raiz da trie $T$ -- quando claro a partir do contexto, será escrito apenas $r$. 

    Uma consulta a uma trie $T$ é representada por $T(r, p)$, $p$ palavra, e tem seu significado definido de maneira recursiva: $T(v, \epsilon)$ corresponde a lista de índices de $v$, e $T(v, aw)$ é $T(u, w)$, se em $v$ há o par $<a, u>$, e $\emptyset$, caso contrário.

    Inserir o par $<p, i>$ em uma trie $T$, onde $p$ é uma palavra e $i$ é um índice, corresponde a criar as estruturas necessárias para que, se antes dessa inserção $T(r, p) = L$, após ela $T(r, p) = L \cup i$.

    \subsection{Manutenção de $N$ chaves de maior frequência} \label{sec:list}
    Para manutenção das $N$ chaves com maior frequência, escolheu-se por utilizar uma lista implementada em vetor com $N$ pares chave-frequência.

    Dadas chaves e suas fre\-quên\-cias, mantém-se os $N$ pares chave-frequência ordenados decrescentemente por valor de frequência.
    A operação de inserção (todas as referências à operação de inserção em listas desse tipo se referirão à operação aqui descrita) de um par chave-frequência pode ser feita $O(1)$ operações, desde que seja feita no fim do vetor subjacente.
    Para manter as chaves em ordem decrescente de frequência e manter apenas pares com valores de chave distintos, a lista é reordenada em $O(N)$ operações e, caso o número de pares supere $N$, o que apresenta menor frequência pode ser descartado em $O(1)$ operações.
    Portanto, a operação de inserção demanda $O(N)$ operações.

    A vantagem em utilizar essa estrutura é que a sua implementação é extremamente simples, é possível acessar sequencialmente os pares de maneira linear e as constantes necessárias para realizar essas operações são pequenas.
    A desvantagem aparenta estar na complexidade linear na operação de inserção de um par.

    Contudo, é importante perceber que o número de operações tem complexidade linear para o valor de $N$ pares \emph{guardados na lista}; ou seja, para um valor fixo de $N$, o custo por operação de inserção é constante.
    Apesar de esse mesmo argumento poder ser feito para qualquer estrutura de dados que guardasse exatamente $N$ pares e descartasse os últimos, como para esse sistema serão necessários apenas listas de tamanho 3, as vantagens da estrutura (acesso linear aos elementos ordenados, constantes por operação baixa e implementação ingênua) superam os benefícios de outras estruturas de dados, como heaps ou árvores balanceadas.

  \section{Detalhes da implementação}
    Cada vez que uma palavra $p$ é inserida na base de palavras, seja via uma operação de inserção, seja via uma operação de digitação, é atribuido a ela um identificador númerico inteiro $i_p$ no intervalo $[0, n - 1]$.
    Em particular, se antes da inserção de $p$ existiam $n$ palavras na base, $i_p \leftarrow n - 1$ e, em seguida, o valor de $n$ aumenta em uma unidade.

    Os identificadores são atribuídos sequencialmente às palavras -- se uma palavra $p$ foi digitada antes de uma palavra $s$, então necessariamente $i_p < i_s$.
    Disso decorre que $i_p$ identifica unicamente a palavra $p$, já que a operação de digitação não permite a reinserção de palavras na base e é um pressuposto que não serão inseridas palavras repetidas através da operação de inserção.

    São mantidos, em memória principal, as $n$ palavras do banco de palavras e suas frequências, duas tries, $T$ e $T'$, o índice da última palavra digitada e, para cada palavra, os índices das três palavras que mais frequentemente a sucederam, guardados em listas como descritos na Seção~\ref{sec:list} -- os índices das palavras representam as chaves das listas.

    \subsection{Persistência}
    Mediante a inserção de $p$ na base de palavras, são criados arquivos ``$i_p$\emph{-word.dat}'' (doravante chamado de arquivo de $p$) e ``$i_p$\emph{-freq.dat}'' (doravante chamado arquivo de frequência de $p$) que guardam, respectivamente, a palavra $p$ e o número de vezes que $p$ foi digitada.
    Quando uma palavra $p$ sucede uma palavra $s$, caso essa seja a primeira ocorrência desse evento, é criado um arquivo ``$i_s$-$i_p$\emph{-freq.dat}'' com valor 1 (esse arquivo será referido como arquivo de $s$-$p$).

    Assim, toda vez que uma palavra $p$ for digitada, sua frequência é lida a partir do seu arquivo de frequência, é incrementada e então é reescrita, agora atualizada, no dispositivo de memória secundária.
    Caso $p$ tenha sucedido uma palavra $s$, é feita uma tentativa de leitura do arquivo de $s$-$p$.
    Se essa tentativa falhar, supõe-se que o arquivo não existe e, como visto anteriormente, o evento ``$p$ sucede $s$'' ainda não havia ocorrido.
    Caso a tentativa seja bem sucedida, a frequência relativa é atualizada de maneira análoga a atualização da frequência absoluta.

    A atribuição de um identificador único e a criação de arquivos individuais permite a atualização das frequências na memória secundária com operações que, descontando o tempo de acesso do arquivo, equivalem a acesso aleatório.
    Essa abordagem é superior a métodos de hashing extensível porque permite recuperar os dados acessando exatamente uma página.

    No começo de cada execução do sistema, as palavras e seus identificadores são carregados na memória principal através da leitura sequencial dos $n$ arquivos de palavra existentes.
    Quando da leitura do arquivo da palavra $p$, seu arquivo de frequência e, para todo $i_s < i_p$, os arquivos $s$-$p$ e $p$-$s$ serão também carregados na memória principal.
    Isso é suficiente para garantir a persistência dos dados.

    \subsection{Consulta de palavras}
    Para consultar se uma determinada palavra está ou não presente na base de palavras de maneira eficiente, é mantida uma trie $T$.
    Quando uma palavra $p$ é inserida no banco de palavras ou carregada da memória secundária, inserimos $<p, i_p>$ em $T$.
    Para determinar se $p$ é uma palavra correta, basta verificar se $T(r, p)~\ne~\emptyset$.
    Com isso, pode-se determinar se uma palavra está ou não presente no banco de palavras em complexidade $O(m)$.

    \subsection{Sugestão de próximas palavras}
    Uma vez digitada a palavra $p$, é possível obter que palavras mais frequentemente a sucederam até a execução do programa até esse ponto acessando a lista disponível em memória principal para esse fim.
    Nela, os elementos já se encontram apropriadamente ordenados.
    Essa consulta, então, pode ser respondida em $O(1)$ operações.

    Para manter as listas de frequências relativas atualizadas, se $p$ sucedeu $s$, é realizada uma operação de inserção na lista que mantém a frequência relativa de $s$-$p$.
    Pela estrutura da lista, isso é suficiente para que, na próxima requisição, as listas apresentem a organização desejada.
    A lista, portanto, também pode ser atualizada em tempo constante.

    \subsection{Possíveis correções}
    Dada uma palavra $p$, as possíveis correções de $p$ foram definidas como palavras que estão no banco de palavras, tem o mesmo comprimento de $p$ e dela diferem pela substituição de um caractere.
    Definamos o conjunto $$\Phi(p)~=~\{uw~\mid~\exists~a \text{ caractere, }~uaw~=~p\}$$
    É direto perceber que uma palavra $s$ é uma possível correção de $p$ se, e somente se, $\Phi(s) \cap \Phi(p) \ne \emptyset$.

    De posse dessa observação, utiliza-se uma segunda trie $T'$ para responder eficientemente às consultas sobre possíveis correções.
    Toda vez que uma palavra $p$ é inserida no banco de palavras ou carregada da memória secundária, inserimos cada palavra em $\Phi(p)$, acompanhada de $i_p$, em $T'$.
    Cada palavra em $\Phi(p)$ possui comprimento de, no máximo, $m - 1$ caracteres -- inserí-las em $T'$ pode ser feito com $O(m^2)$ operações.
    Para obter as possíveis correções de uma palavra $p$, então, computamos $\bigcup_{s \in \Phi(p)} T'(r, s)$ em $O(m^2)$.
    Pode-se inserir os pares índice-frequência em uma lista como a descrita na Seção~\ref{sec:list} com $O(n)$ operações; os três elementos presentes na lista após todas as inserções são as correções desejadas.
    A complexidade para encontrar as 3 possíveis correções mais frequentes é, portanto, $O(m^2 + n)$.

    \subsection{Impressão de frequências}
    Para imprimir as palavras e suas frequências em ordem alfabética, percorremos a trie $T$ em ordem infixa e guardamos, na ordem em que são vistos, os índices em seus nós.
    O custo computacional dessa operação é igual ao número de nós na trie $T$: $O(nm)$.
    Esse procedimento coloca os nós em ordem alfabética na lista retornada. 
    Vale lembrar que as frequências com que as palavras aparecem estão armazenados na memória principal.

    A consulta de quais palavras mais sucederam uma dada palavra é trivialmente respondida pela lista mantida em memória principal.

\end{document}
